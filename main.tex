\documentclass[10pt]{article}
\usepackage{NotesTeXV3,lipsum}
\usepackage[spanish, es-tabla]{babel}
\usepackage{cancel} 
\usepackage{subfiles}
\graphicspath{{\subfix{images/}}}
%\usepackage{showframe}

\begin{document}
	\title{{Mecánica Clásica}\\{\normalsize{\itshape Notas a un nivel intermedio}}}
	\author{Elmer Hernán Barquero Chaves}
	\affiliation{
	Estudiante de bachillerato en física por la Universidad de Costa Rica\\
	%\href{https://adhumunt.github.io/}{Website}\\
	%\href{https://inspirehep.net/authors/1669979}{Inspire-HEP}\\
	%\href{https://www.linkedin.com/in/aditya-dhumuntarao-80a0b8112/}{LinkedIn}\\
	\href{https://github.com/201Hg}{GitHub}\\
	Estas notas son un documento en proceso por lo que es posible que contengan errores, los cuales agradecería informaran al correo, de preferencia con un asunto: `` Notas de mecánica''  o algún similar. Para realizar la corrección lo más pronto posible.
	}
	\emailAdd{elmer.barquero@ucr.ac.cr}
	\maketitle
	\newpage
	\pagestyle{fancynotes}



	\part{Introducción al texto}
	\label{part: intro}

	\subfile{chapters/cap0.tex}

	\part{Mecánica Newtoniana} % (fold)
	\label{prt:Mecánica Newtoniana}

	\subfile{chapters/cap1.tex}

	
	% part Mecánica Newtoniana (end)



	%%%%%%%%%%%%%%%%%%%%%%%%%%%%%%%
	% bibliografía

	% Se usó ese código para generar nota en el índice y que aparezcan todos documentos que forman el archivo .bib de forma automatica sin previa cita en el text
	%%%%%%%%%%%%%%%%%%%%%%%%%%%%%%%
	\phantomsection

	\addcontentsline{toc}{section}{Referencias}
	\bibliographystyle{unsrtnat} %puede cambiarse, de ser necesario
	\bibliography{biblio.bib}
	\cite{*} %quitarlo cuando realizan las citas correspondientes

\end{document}