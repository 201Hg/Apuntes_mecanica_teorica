\documentclass[/home/hernan/Documentos/Apuntes_mecanica_teorica/main.tex]{subfiles}
\graphicspath{{\subfix{images/}}}

\begin{document}
    \part{Mecánica Lagrangiana}

    Este capítulo se va a dedicar a establecer un formalismo alternativo a las Leyes de Newton, el formalismo lagrangiano aplicado a la mećanica clásica. Este nuevo formalismo produce resultados equivalentes al de Newton, como es de esperarse, no obstante permite simplificar de gran manera los análisis.

    Aquí se expresará la Mecánica Lagrangiana de forma general, de modo que se aplique a sistemas conservativos (o no) y a sistemas de partículas.

    \section{Conceptos Fundamentales}

    \subsection{Grados de Libertad y Coordenadas Generalizadas}

    \begin{definition}[\textbf{Grados de Libertad}]
        Para un sistema físico cualquiera, el número de grados de libertad corresponde al número más pequeño de cantidades escalares independientes necesarias para dar la posición (Sin contar el tiempo) de un objeto de interés dentro del sistema.

        \begin{itemize}
            \item \textbf{Partícula en una dimensión:} 1 grado de libertad, la partícula se encuentra sobre una recta está puede ir hacia adelante o atrás.
            \item \textbf{Partícula en 2 dimensiones:} 2 grados de libertad, la partícula tiene libertad de moverse en un plano.
            \item \textbf{Partícula en 3 dimensiones:} 3 grados de libertad, la partícula tiene puede moverse en un espacio tridimensional (en un ancho, largo y altura respecto a un origen).
            \item \textbf{n-Partículas en 3 dimensiones:} 3n grados de libertad, cada partícula individual tiene sus 3 grados de libertada propios y la suma es el total de grados de libertad del sistema.
        \end{itemize}
        
    \end{definition}

    \begin{definition}[\textbf{Fuerzas de Restricción}]
        Las fuerzas de restricción, como su nombre lo indican, restringen el movimiento que puede tener un sistema desde la libertad total, es decir, cortan los grados de libertad que posee un sistema. Algunos ejemplos:
        \begin{itemize}
            \item La tensión debido a la cuerda en un péndulo.
            \item La fuerza normal.
            \item Fuerzas electromagnéticas entre partículas de un sólido.
        \end{itemize}
    \end{definition}

    \begin{definition}[\textbf{Restricciones}]
        Las restricciones son las limitaciones que se le imponen al movimiento del sistema, ya sea por observación de dicha limitación o simplificación del sistema y son dadas por medios de ecuaciones que contengan los parámetros que son restringidos en el sistema (Posiciones, velocidades, ...). Las restricciones que posea un sistema están directamente relacionadas a las fuerzas de restricción que el sistema siente, de acuerdo a los ejemplos de fuerzas de restricción:

        \begin{itemize}
            \item Un sistema atado a una cuerda.
            \item Un sistema sobre una o entre superfies.
            \item Un cuerpo, sus partículas se mueven muy poco o no lo hacen en sus posiciones relativas al CM.
        \end{itemize}
        
        Existen diferentes tipos de restricciones de acuerdo a como se expresan:

        \begin{itemize}
            \item \textcolor{red}{Holonómicas:} Son restricciones que se pueden escribir de la formar: 
                \begin{equation}
                    g\left(\vec{r}_{1},\vec{r}_{2}, ...,\vec{r}_{n}\right) = 0
                \end{equation}
            \item \textcolor{red}{No Holonómicas:} Son aquellas restricciones que no pueden ser escritas como las holonómicas.
        \end{itemize}

        También es posible clasificar a las restricciones a partir de su dependencia temporal:

        \begin{itemize}
            \item \textcolor{red}{Esclerónomas:} El tiempo no es una variable explícita
            \item \textcolor{red}{Reónomas:} El tiempo es una variable explícita
        \end{itemize}

    \end{definition}

    A partir de aquí se va a trabajar con restricciones holonómicas únicamente a menos de que se indique lo contrario. Esto porque este tipo de restricción permite reducir el número de coordenadas necesarias para describir el movimiento de un sistema a partir de un nuevo conjunto de coordenadas conocido como coordenadas generalizadas.

    \begin{definition}[\textbf{Coordenadas Generalizadas}]
        Son el conjunto de cantidades \mn{No necesariamente tienen que ser distancias, pueden ser cantidades adimensionales o incluso con unidades de energía.} que especifican \textcolor{red}{completamente} el estado de un sistema. \\

        Ahora, suponiendo que un sistema de estudio está conformado por $n$ partículas y al estar en un espacio tridimensional, en un principio el sistema posee $3n$ grados de libertad. Si el sistema posee \textcolor{orange}{$m$ ecuaciones  de restricción}\mn{Asociadas o no a fuerzas de restricción. Estas pueden ser incluso cortes en la dimensionalidad del espacio en que se encuentra el sistema.}, tendrá $m$ restricciones en los grados de libertad totales del sistema, entonces los grados de libertad son:

        \begin{equation*}
            s = 3n -m \leftarrow     \begin{matrix} \textup{Grados de libertad de un} \\ \textup{sistema en forma general}\end{matrix}       
        \end{equation*}

        Es posible escribir las ecuaciones de transformación entre las coordenadas generalizadas y coordenadas curvilíneas cualesquiera, como se presenta a continuación:

        \begin{equation*}
            \begin{matrix}
                \begin{matrix}
                \textup{Coordenadas Curvilíneas}  \\ 
                    \\ 
                    \left\{\begin{matrix}
                            x_{\alpha,i} = x_{\alpha,i}\left ( q_{j},t \right ) \\ 
                            \dot{x}_{\alpha,i} = \dot{x}_{\alpha,i} \left (  q_{j}, \dot{q}_{j},t \right ) \\ 
                            \end{matrix}\right .
                \end{matrix} 
                
                \leftrightarrow 
                
                \begin{matrix}
                    \textup{Coordenadas Generalizadas} \\ 
                    \\ 
                    \left\{\begin{matrix}
                                q_{j} = q_{j} \left ( x_{\alpha,i}, t \right )\\ 
                                \dot{q}_{j} = \dot{q}_{j} \left ( x_{\alpha,i}, \dot{x}_{\alpha,i} , t\right )
                            \end{matrix}\right .
                
                \end{matrix} 
                \\ 
                \\ 
                \left\{\begin{matrix}
                            \alpha = 1,2,...,n\\ 
                            i= 1,2,3\\ 
                            j= 1,2,...s
                        \end{matrix}\right .
            \end{matrix}
        \end{equation*}

        Si se tiene el conjunto más pequeño de coordenadas generalizadas, este es conocido como el \textcolor{blue}{conjunto adecuado} de coordenadas \mn{Va a ser común desarrollar un problemas tanto con el conjunto adecuado como con un conjunto de cantidades que superen al adecuado.}. En este caso, el conjunto de cantidades (Coordenadas Generalizadas) corresponde a los grados de libertad del sistema.
    \end{definition}

    \subsection{Trabajo Virtual, Fuerzas Generalizadas y Principio de D'Alambert}

    A partir de aquí se considerará un sistema compuesto por $n$ partículas, cada una de ellas expuesta a su vector de fuerza neta correspondiente $F_{\alpha}$. Siendo $F_{\alpha}$ la suma de cualquier clase de fuerzas que actuen sobre la partícula $\alpha$ (Fuerzas conservativas,  no conservativas, de restricción) y posee sus componentes, por ejemplo, en coordenadas cartesinas $F_{\alpha_{x}}$ , $F_{\alpha_{y}}$ y $F_{\alpha_{z}}$. Además, la fuerza se puede separar de la forma $\vec{F}_{\alpha} = \vec{F}_{\alpha} ^{e} + \vec{f}_{\alpha} $, donde $\vec{F}_{\alpha} ^{e}$ corresponde a todas las fuerzas externas aplicadas en la partícula $\alpha$ y $\vec{f}_{\alpha}$ corresponde a todas las fuerzas de restricción que afectan a dicha partícula.

    Un trabajo infinitesimal provocado por esta fuerza:

    \begin{align*}
        \delta W &= \sum_{\alpha=1}^{n} \vec{F}_{i}  \cdot \delta \vec{r}_{\alpha} \\ 
                &= \sum_{\alpha=1}^{n} \vec{F}_{\alpha} ^{e}  \cdot \delta \vec{r}_{\alpha} + \sum_{\alpha=1}^{n} \vec{f}_{\alpha}  \cdot \delta \vec{r}_{\alpha}
    \end{align*}

    \textcolor{red}{Limitando el análisis a sistemas tales que las fuerzas de restricción no producen este tipo de trabajo infinitesimal:} 

    \begin{align*}
        \delta W &= \sum_{\alpha=1}^{n} \vec{F}_{\alpha} ^{e} \cdot \delta \vec{r}_{\alpha} + \cancelto{0}{\sum_{\alpha=1}^{n} \vec{f}_{\alpha}  \cdot \delta \vec{r}_{\alpha}} \\ 
            & = \sum_{\alpha=1}^{n} \vec{F}_{\alpha} ^{e} \cdot \delta \vec{r}_{\alpha}
    \end{align*}

    Redefiniendo el vector $\vec{F}_{\alpha}$ como la suma de todas las fuerzas aplicadas sobre la partícula $\alpha$, excepto las de restricción, dado que ya se establecio que para este análisis estas no producen trabajo. se introduce el siguiente concepto:


    \begin{definition}[\textbf{Trabajo Virtual}] 
        Corresponde al trabajo producido por un \textcolor{blue}{desplazamiento virtual \mn{Al decir ``virtual'', es para diferenciarlo de un desplazamiento o trabajo real del sistema.}}. Un desplazamiento virtual es un desplazamiento infinitesimal de un sistema, una alteración en la configuración de este, como resultado de un cambio infinitesimal arbitrario de las coordenadas $\delta \vec{r}_{\alpha}$ , que debe ser consistente con las fuerzas y las restricciones impuestas en el sistema en un cierto instante $t$. \\ 

        \begin{equation}
            \delta W = \sum_{\alpha=1}^{n} \vec{F}_{\alpha}  \cdot \delta \vec{r}_{\alpha}
            \label{eq: trabajovirtual}
        \end{equation}
    \end{definition}

    A partir de la definición anterior, buscando colocar el trabajo virtual en el conjunto de coordenadas generalizadas:\\ 

    Es posible desarrollar $ \delta \vec{r}_{\alpha}$ a partir de la regla de la cadena para $3$ grados de libertad como:

    \begin{equation*}
        \delta \vec{r}_{\alpha} = \sum_{j=1}^{s} \frac{\partial \vec{r}_{\alpha}}{\partial q_{j}} \delta q_{j}
    \end{equation*}

    Excluyendo la derivada parcial temporal por la definición del desplazamiento virtual, que solo considera desplazamientos en las coordenadas. Realizando el cambio de $\delta \vec{r}_{\alpha}$  en la \Eqref{eq: trabajovirtual}:

    \begin{align*}
        \delta W & = \sum_{\alpha=1}^{n} \vec{F}_{\alpha}  \cdot \delta \vec{r}_{\alpha} \\ 
                & = \sum_{\alpha=1}^{n} \vec{F}_{\alpha}  \cdot \sum_{j=1}^{s} \frac{\partial \vec{r}_{\alpha}}{\partial q_{j}} \delta q_{j} \\ 
                & = \sum_{\alpha=1}^{n} \sum_{j=1}^{s} \vec{F}_{\alpha} \cdot \frac{\partial \vec{r}_{\alpha}}{\partial q_{j}} \delta q_{j} \\ 
                & = \sum_{j=1}^{s} Q_{j} \delta q_{j}
    \end{align*}

    De lo anterior se obtiene:

    \begin{equation*}
        Q_{j} = \sum_{\alpha=1}^{n} \vec{F}_{\alpha} \cdot \frac{\partial \vec{r}_{\alpha}}{\partial q_{j}} 
    \end{equation*}

    Lo cual al considerar que tanto $\vec{F}_{\alpha}$ y $\vec{r}_{\alpha}$ se encuentran inicialmente en coordenadas curvilíneas:

    \begin{equation*}
        Q_{j} = \sum_{\alpha=1}^{n} \left( F_{x_{1,\alpha}} \frac{\partial x_{1,\alpha}}{\partial q_{j}} +   F_{x_{2,\alpha}} \frac{\partial x_{2,\alpha}}{\partial q_{j}} + F_{x_{3,\alpha}} \frac{\partial x_{3,\alpha}}{\partial q_{j}} \right)
    \end{equation*}

    \begin{definition}[\textbf{Fuerzas Generalizadas}] 
        Corresponde a un nombre genérico para referirse, en un principio, a fuerzas y torques escritos en las coordenadas generalizadas que describen el sistema. Estas son dadas en los componentes correspondientes a cada coordenada generalizada. Comúnmente las fuerzas o torques son escritas inicialmente en coordendas curvilíneas en un inicio para luego ser transformadas a fuerzas generalizadas.

        \begin{equation}
            Q_{j} = \sum_{\alpha=1}^{n} \left( F_{x_{1,\alpha}} \frac{\partial x_{1,\alpha}}{\partial q_{j}} +   F_{x_{2,\alpha}} \frac{\partial x_{2,\alpha}}{\partial q_{j}} + F_{x_{3,\alpha}} \frac{\partial x_{3,\alpha}}{\partial q_{j}} \right)
            \label{eq: fuerzasgeneralizadas}
        \end{equation}

        Para conocer la naturaleza de la fuerza generalizada (Fuerza, Torque, ...), se debe revisar la dimensionalidad del trabajo virtual ejercido por tal fuerza, es decir:

        \begin{equation*}
            \delta W_{j} = Q_{j} \delta q_{j} \; \; \textup{; Debe poseer unidades de energía}
        \end{equation*}

        Una vez establecidas las coordenadas generalizadas, es sencillo describir a que corresponden las fuerzas generalizadas a partir de lo anterior.
        
    \end{definition}
    
    Ahora, en busca de generar una forma alternativa de mecánica, se va a explotar el concepto anterior. Primero, a partir de la \Eqref{eq: NSecondlaw}:

    \begin{align*}
        \vec{F}_{\alpha} &= \dot{\vec{p}}_{\alpha} \\ 
        \Rightarrow \vec{F}_{\alpha} &- \dot{\vec{p}}_{\alpha} = 0 \\ 
    \end{align*}

    Esta expresión enuncia que una partícula expuesta a una fuerza $ \vec{F}_{\alpha}$, se encontrará en equilibrio si se le es ejercida una fuerza efectiva contraria a la original $- \dot{\vec{p}}_{\alpha}$. Entonces, con esto se puede buscar el trabajo virtual de todas estas fuerzas sobre el sistema, que por lo anterior se sabe que será cero.

    \begin{align*}
        \delta W = \sum_{\alpha=1}^{n} \left( \vec{F}_{\alpha} - \dot{\vec{p}}_{\alpha} \right) \cdot \delta \vec{r}_{\alpha} = 0
    \end{align*}

    \begin{definition}[\textbf{Principio de D'Alambert}] 
        Este principio esta expresando en coordenadas generalizadas.

        \begin{equation}
            \sum_{\alpha=1}^{n} \left( \vec{F}_{\alpha} - \dot{\vec{p}}_{\alpha}\right) \cdot \delta \vec{r}_{\alpha} = 0
            \label{eq: DAlambertP}
        \end{equation}
        
    \end{definition}

    Un resultado de este principio, es el Principio de Trabajo Virtual \mn{Es el caso estático del Principio de D'Alambert, ampliamente usado en ingeniería.}:

        \begin{equation*}
            \delta W = \sum_{\alpha=1}^{n} \vec{F}_{\alpha}  \cdot \delta \vec{r}_{\alpha} = 0
        \end{equation*}



    \section{Principio de Hamilton}

    Este principio puede ser deducido desde el Principio de D'Alambert, no obstante, la deducción no se realizará por ahora.

    \begin{definition}[\textbf{Principio de Hamilton Extendido}]
        El movimiento de un sistema desde un tiempo $t_1$ a un tiempo $t_2$ es tal que la integral de línea (La llamada Acción o integral de la Acción) de la energía cinética más el trabajo ejercido por las fuerzas del sistema, tenga un valor estacionario para el camino real que sigue el sistema \mn{El Principio de Hamilton original se define de forma similar, donde la Acción es de la forma: \begin{equation*} I = \int_{t_1}^{t_2} T - V dt \end{equation*}}.

        \begin{equation}
            I = \int_{t_1}^{t_2} T\left(q_{j}, \dot{q}_{j}, t\right) + W dt 
            \label{eq: PHamiltonE}
        \end{equation}

        \begin{equation*}
            \delta \int_{t_1}^{t_2} T\left(q_{j}, \dot{q}_{j}, t\right) + W dt = 0 \leftarrow \textup{Valor estacionario}
        \end{equation*}

    \end{definition}

    \section{Ecuaciones de la Mecánica de Lagrange}

    A continuación, en esta sección se va deducir la \textcolor{red}{ecuación de Euler - Lagrange}  a partir del Principio de Hamilton Extendido de modo que sea aplicable a sistemas no conservativos y se introducirán conceptos propios de la Mecánica Lagrangiana. \\ 

    \subsection{Deducción del formalismo Lagrangiano}

    Considere un sistema compuesto por $n$ partículas y con $m$ restricciones, dando a lugar $s = n-m$ grados de libertad, igual a la cantidad de coordenadas generalizadas ($q_{1},q_{2},...,q_{s}$) que describen la dinámica del sistema. Además, se expone a cada una de las partículas a una fuerza neta (totalmente conservativa, no conservativa o con ambos tipos de fuerzas) $Q_{j}$ con $j=1,2,...s$.

    Comenzando con la deducción de la ecuación Euler - Lagrange, se debe optimizar la Acción (\Eqref{eq: PHamiltonE}) para conseguir un valor estacionario, hay que tomar la variación de esta e igualarla a cero:
    
    \begin{equation*}
        \delta I = \delta \int_{t_1}^{t_2} T\left(q_{j}, \dot{q}_{j}, t\right) + W dt = 0 
    \end{equation*}

    \begin{align*}
        & \delta \int_{t_1}^{t_2} T\left(q_{j}, \dot{q}_{j}, t\right) + W dt = 0 \\ 
        \Rightarrow & \int_{t_1}^{t_2} \delta T\left(q_{j}, \dot{q}_{j}, t\right) + \delta  W dt = 0 \\ 
        \Rightarrow  \int_{t_1}^{t_2}  & \left[  \sum_{j=1}^{s} \left( \frac{\partial T}{\partial q_{j}}   \delta q_{j} +\frac{\partial T}{\partial \dot{q}_{j}} \delta \dot{q}_{j}  \right) + \sum_{j=1}^{s} Q_{j} \delta q_{j}  \right] dt = 0 \\ 
        \Rightarrow \int_{t_1}^{t_2} \sum_{j=1}^{s} & \left(\frac{\partial T}{\partial q_{j}} \delta q_{j} + \textcolor{blue}{\frac{\partial T}{\partial \dot{q}_{j}} \delta \dot{q}_{j}}   + Q_{j} \delta q_{j} \right) dt = 0 \\ 
    \end{align*}

    Con este resultado, ahora se tomará el terminó marcado en azul para transformarlo en una expresión equivalente:

    \begin{align*}
        \int_{t_1}^{t_2} \sum_{j=1}^{s} \frac{\partial T}{\partial \dot{q}_{j}} \delta \dot{q}_{j} dt & = \int_{t_1}^{t_2} \sum_{j=1}^{s} \frac{\partial T}{\partial \dot{q}_{j}} \delta \left(\frac{d q_{j}}{d t}\right) dt \\ 
        & = \int_{t_1}^{t_2} \sum_{j=1}^{s} \frac{\partial T}{\partial \dot{q}_{j}} \frac{d}{d t} \left(\delta  q_{j}\right) dt \\ 
        & = \sum_{j=1}^{s} \int_{t_1}^{t_2} \frac{\partial T}{\partial \dot{q}_{j}} \frac{d}{d t} \left(\delta  q_{j}\right) dt \\ 
    \end{align*}

    Resolviendo esta integral por integración por partes para un término arbitrario de la suma \mn{El término: \begin{equation*} \left.  \frac{\partial T}{\partial \dot{q}_{j}} \delta  q_{j}  \right|_{t_1}^{t_2} = 0 \end{equation*} porque los extremos están fijos, de modo que la variación en la coordenada generalizada $q_{j}$ es igual a $0$ cuando se está en el momento inicial $t_1$ y en el momento final $t_2$ }:

    \begin{align*}
        \int_{t_1}^{t_2} \overbrace{\frac{\partial T}{\partial \dot{q}_{j}}}^{\textcolor{orange}{u}} \underbrace{\frac{d}{d t} \left(\delta  q_{j}\right)}_{\textcolor{violet}{dv} } dt & = \cancelto{0}{\left.  \frac{\partial T}{\partial \dot{q}_{j}} \delta  q_{j}  \right|_{t_1}^{t_2}} - \int_{t_1}^{t_2}  \delta  q_{j} \frac{d}{d t} \left(\frac{\partial T}{\partial \dot{q}_{j}} \right) dt =  - \int_{t_1}^{t_2}  \frac{d}{d t} \left(\frac{\partial T}{\partial \dot{q}_{j}} \right)  \delta  q_{j}  \; dt \\ 
    \end{align*}

    De volviendo esta expresión al interior de la ecuación inicial:

    \begin{align*}
        & \int_{t_1}^{t_2} \sum_{j=1}^{s}  \left(\frac{\partial T}{\partial q_{j}} \delta q_{j} + \textcolor{blue}{\frac{\partial T}{\partial \dot{q}_{j}} \delta \dot{q}_{j}}   + Q_{j} \delta q_{j} \right) dt = 0 \\ 
        \Rightarrow  & \int_{t_1}^{t_2} \sum_{j=1}^{s}  \left(\frac{\partial T}{\partial q_{j}} \delta q_{j} + \textcolor{blue}{- \frac{d}{d t} \left(\frac{\partial T}{\partial \dot{q}_{j}} \right)  \delta  q_{j}}   + Q_{j} \delta q_{j} \right) dt = 0 \\ 
        \Rightarrow  & \int_{t_1}^{t_2} \left \{ \sum_{j=1}^{s}  \left[\frac{\partial T}{\partial q_{j}}  - \frac{d}{d t} \left(\frac{\partial T}{\partial \dot{q}_{j}} \right)     + Q_{j}  \right] \delta q_{j} \right\}  dt = 0
    \end{align*}

    Entonces, para que esta integral sea igual a cero, el contenido de la integral debe ser igual a cero para valores arbitrarios de las variaciones $\delta q_{1}$ , $\delta q_{2}$ , ... , $\delta q_{s}$: 

    \begin{equation*}
        \sum_{j=1}^{s}  \left[\frac{\partial T}{\partial q_{j}}  - \frac{d}{d t} \left(\frac{\partial T}{\partial \dot{q}_{j}} \right)     + Q_{j}  \right] \delta q_{j} = 0
    \end{equation*}

    Dada la arbitrariedad de dichas variaciones, es posible suponer que todas las variaciones de $\delta q$'s son cero, excepto un $\delta q_{j}$. Se debe cumplir para ese $\delta q_{j}$:

    \begin{equation*}
        \frac{\partial T}{\partial q_{j}}  - \frac{d}{d t} \left(\frac{\partial T}{\partial \dot{q}_{j}} \right)     + Q_{j} = 0
    \end{equation*}

    Reescribiendo \mn{Esta última ecuación ya posee la forma de una ecuación de Euler}:

    \begin{equation*}
        \frac{d}{d t} \left(\frac{\partial T}{\partial \dot{q}_{j}} \right) - \frac{\partial T}{\partial q_{j}} = Q_{j}
    \end{equation*}

    Para terminar esta deducción, es necesario separar las fuerzas conservativas (c) de las no conservativas (nc) \marginnote{\textcolor{red}{Recuerde que $Q_{j}$ no posee ninguna fuerza de restricción}}, entonces:

    \begin{align*}
        Q_{j} & = Q_{j}^{c} + Q_{j}^{nc} \\ 
            & = \sum_{\alpha}^{n} \vec{F}_{\alpha}^{c} \cdot \frac{\partial \vec{r}_{\alpha}}{\partial q_{j}} + \sum_{\alpha}^{n} \vec{F}_{\alpha}^{nc} \cdot \frac{\partial \vec{r}_{\alpha}}{\partial q_{j}}
    \end{align*}

    Ahora, es necesario recordar la \Eqref{eq: conservativeforce} para fuerzas conservativas, lo que permite: 

    \begin{align*}
        Q_{j} & = \sum_{\alpha}^{n} - \vec{\nabla} V_{\alpha} \cdot \frac{\partial \vec{r}_{\alpha}}{\partial q_{j}} + \sum_{\alpha}^{n} \vec{F}_{\alpha}^{nc} \cdot \frac{\partial \vec{r}_{\alpha}}{\partial q_{j}} \\ 
        & =  - \sum_{\alpha}^{n} \left( \frac{\partial V_{\alpha}}{\partial x_{1,\alpha}} \frac{\partial x_{1,\alpha}}{\partial q_{j}} +   \frac{\partial V_{\alpha}}{\partial x_{2,\alpha}} \frac{\partial x_{2,\alpha}}{\partial q_{j}} + \frac{\partial V_{\alpha}}{\partial x_{3,\alpha}} \frac{\partial x_{3,\alpha}}{\partial q_{j}} \right) + Q_{j}^{nc} \\ 
        & = - \sum_{\alpha}^{n} \frac{\partial V_{\alpha}}{\partial q_{j}} + Q_{j}^{nc} \\ 
        & = - \frac{\partial V}{\partial q_{j}} + Q_{j}^{nc}
    \end{align*}

    El volver a colocar $ \vec{F}^{nc}$ como $ Q_{j}^{nc}$ no representa un problema por ahora, debido a que la forma de estas fuerzas puede llegar a depender del problema en que se planteen y no hay ninguna pérdida al dejarlas de forma general. \\ 

    Con la expresión anterior, se introduce en la ecuación de Euler:

    \begin{align*}
        & \frac{d}{d t} \left(\frac{\partial T}{\partial \dot{q}_{j}} \right) - \frac{\partial T}{\partial q_{j}} = Q_{j} \\ 
        \\
        \Rightarrow & \frac{d}{d t} \left(\frac{\partial T}{\partial \dot{q}_{j}} \right) - \frac{\partial T}{\partial q_{j}} = - \frac{\partial V}{\partial q_{j}} + Q_{j}^{nc} \\ 
        \\
        \Rightarrow & \frac{d}{d t} \left(\frac{\partial T}{\partial \dot{q}_{j}} \right) - \left(\frac{\partial T}{\partial q_{j}} - \frac{\partial V}{\partial q_{j}}\right) =  Q_{j}^{nc}
    \end{align*}

    A esta última ecuación no le hará ningún daño aprovechar encarecidamente de la definición que acompaña a la \Eqref{eq: conservativeforce}, diciendo lo siguiente \mn{Introducir el término \begin{equation*} -\frac{d}{d t} \left(\frac{\partial V}{\partial \dot{q}_{j}}\right)  \end{equation*} no debe ser un problema, puesto que en un inicio será cero por la limitación de fuerzas conservativas. \\ No obstante, fuera de la mecánica clásica será común que este término no sea cero en un intento de escribir potenciales para fuerzas que, al menos en el sentido usual, no los poseen. Por ejemplo la fuerza de Lorentz que da lugar a potenciales generalizados $V = V\left(q_{j}, \dot{q}_{j}\right) $.}:

    \begin{align*}
        & \frac{d}{d t} \left(\frac{\partial T}{\partial \dot{q}_{j}} \right) \textcolor{blue}{-\frac{d}{d t} \left(\frac{\partial V}{\partial \dot{q}_{j}}\right) } - \left(\frac{\partial T}{\partial q_{j}} - \frac{\partial V}{\partial q_{j}}\right) =  Q_{j}^{nc} \\ 
        \\
        \Rightarrow & \frac{d}{d t} \left(\frac{\partial \left(T - V\right) }{\partial \dot{q}_{j}} \right) - \frac{\partial \left(T - V\right)}{\partial q_{j}} =  Q_{j}^{nc} \\
    \end{align*}

    \subsection{Ecuaciones de Euler-Lagrange}

    \begin{definition}[\textbf{Lagrangiano}]
        Se define la función $\mathcal{L}$ como el Lagrangiano del sistema, función que para el caso de un sistema conservativo dicta toda la dinámica del sistema al operarlo para cada coordenada generalizada en la ecuación de Euler-Lagrange.

        \begin{equation}
            \mathcal{L} = T - V
            \label{eq: Lagrangiano}
        \end{equation}

        \begin{itemize}
            \item $T$ : Es la energía cinética total del sistema, que posea el sistema.
            \item $V$ : Es la energía potencial total del sistema, la suma de todas las energías potenciales que afecten al sistema.
        \end{itemize}
        
    \end{definition}

    \begin{definition}[\textbf{Ecuaciones de Euler-Lagrange para sistemas semi-conservativos}] 
        
        \begin{equation}
            \frac{d}{d t} \left(\frac{\partial \mathcal{L} }{\partial \dot{q}_{j}} \right) - \frac{\partial \mathcal{L}}{\partial q_{j}} =  Q_{j}^{nc} \; \; \; ; \; j=1,2,...,s
            \label{eq: eqlagrange}
        \end{equation}

        Al operar esta ecuación sobre el Lagrangiano del sistema y tomando o no las fuerzas disipativas para cada coordenada generalizada, la solución que genera serán s ecuaciones diferenciales de segundo orden que dan las condiciones para que el camino que sigue el sistema en el espacio de configuración sea un valor estacionario para el Principio de Hamilton (Extendido) \Eqref{eq: PHamiltonE}.
        
    \end{definition}

    \section{Ecuaciones de la Mecánica de Lagrange con Restricciones}

    En esta sección se va a ampliar la mecánica de Lagrange para resolver problemas con condiciones \textcolor{red}{no holonómicas de cierto tipo} \mn{El formalismo será válido para restricciones semi-holonómicas: \begin{equation*} g\left(q_{1}, q_{2}, ...q_{s}, \dot{q}_{1}, \dot{q}_{2}, ..., \dot{q}_{s}, t \right) =0 \end{equation*} } a partir del Principio de Hamilton (Extendido). Para esto se va a comenzar de nuevo con la deducción de las ecuaciones de Euler-Lagrange, pero se van a usar resultados ya mostrados de la deducción anterior que aquí también sean válidos.

    \subsection{Deducción del formalismo de Lagrange con Restricciones}

    Nuevamente considere un sistema compuesto por $n$ partículas y con $m$ restricciones $g_{i} \left(q_{1},q_{2},...,q_{s}, \dot{q}_{1},\dot{q}_{2},...\dot{q}_{s}, t \right) = 0$, dando a lugar $s\geq3n-m$ grados de libertad, se puede hallar un conjunto de \textcolor{blue}{coordenadas generalizadas adecuado ($q_{1},q_{2},...,q_{s}$)} tal que sea \textcolor{blue}{independiende de las restricciones} impuestas al sistema. Además, se expone a cada una de las partículas a una fuerza neta\mn{\textcolor{red}{Recuerde que esta fuerza no posee ningún tipo de fuerza de restricción en su interior, es una agrupación de fuerzas conservativas y no conservativas.}}  $Q_{j}$ con $j=1,2,...s$.\\ 

    Entonces, aplicando al Principio de Hamilton Extendido la técnica de Multiplicadores Indeterminados de Lagrange y suponiendo que el multiplicador es de la forma $\lambda \left(q_{j}, \dot{q}_{j}, t \right)$ , se llega a:

    \begin{align*}
        I = \int_{t_1}^{t_2} \left[ T\left(q_{j}, \dot{q}_{j}, t\right) + W + \sum_{i=1}^{m} \lambda_{i}\left(q_{j}, \dot{q}_{j}, t \right) \; g_{i}\left(q_{j}, \dot{q}_{j}, t \right) \right] dt 
    \end{align*}

    Buscando un punto de estacionario de la Acción, hay que tomar la variación de esta e igualarla a cero:

    \begin{align*}
        & \delta I = \delta  \int_{t_1}^{t_2} \left[ T\left(q_{j}, \dot{q}_{j}, t\right) + W + \sum_{i=1}^{m} \lambda_{i}\left(q_{j}, \dot{q}_{j}, t \right) \; g_{i}\left(q_{j}, \dot{q}_{j}, t \right) \right] dt = 0 \\ 
        \Rightarrow & \int_{t_1}^{t_2} \left[ \delta T\left(q_{j}, \dot{q}_{j}, t\right) + \delta  W + \delta \sum_{i=1}^{m} \lambda_{i}\left(q_{j}, \dot{q}_{j}, t \right) \; g_{i}\left(q_{j}, \dot{q}_{j}, t \right) \right] dt = 0 \\ 
    \end{align*}

    Desarrollando primero los términos ya conocidos, $\delta T$ y $\delta W$:

    \begin{align*}
        \Rightarrow &  \int_{t_1}^{t_2} \left\{ \sum_{j=1}^{s} \left[ \frac{\partial T}{\partial q_{j}}  - \frac{d}{d t} \left(\frac{\partial T}{\partial \dot{q}_{j}} \right) + Q_{j} \right] \delta q_{j}  + \delta \sum_{i=1}^{m} \lambda_{i}\left(q_{j}, \dot{q}_{j}, t \right) \; g_{i}\left(q_{j}, \dot{q}_{j}, t \right)\right\} dt = 0 \\ 
        \Rightarrow & 
    \end{align*}


    \subsection{Ecuaciones de Euler-Lagrange con Restricciones}
    \begin{definition}[\textbf{Ecuaciones de Euler-Lagrange con Restricciones}]
        \begin{equation}
            \frac{d}{d t} \left(\frac{\partial \mathcal{L} }{\partial \dot{q}_{j}} \right) - \frac{\partial \mathcal{L}}{\partial q_{j}} + \sum_{i=1}^{m}\lambda_{i} \frac{\partial g_{i}}{\partial q_{j}}=  Q_{j}^{nc}
        \end{equation}
    \end{definition}


    \subsection{Problemas resueltos}
    %\includepdf[pages={82-105,107-111},templatesize = {8 in}{8.5 in} ]{AMeca.pdf}
%%%%%   

\end{document}