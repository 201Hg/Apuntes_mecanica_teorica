\documentclass[/home/hernan-barquero/Documents/Apuntes_mecanica_teorica/main.tex]{subfiles}
\graphicspath{{\subfix{images/}}}


\begin{document}
    \part{Introducción al texto}
    \label{part: intro}
    
    \section{Amplitud de la Mecánica Clásica}

    %En este inicio de las notas se hablará de la amplitud que posee la Mecánica Clásica para afrontar fenómenos del universo, comentando condiciones que limitan su aplicabilidad y las fronteras que esta posee para producir descripciones confiables de la realidad. 

    Condiciones para la aplicabilidad de la teoría de la Mecánica Clásica:

    \begin{itemize}
        \item Las masas de los cuerpos de interés \textbf{deben ser mayores} a las masas de átomos y de partículas subatómicas.
        \item Las masas de los cuerpos de interés \textbf{deben ser pequeñas} en comparación a las masas de cuerpos celestes. Por ejemplo la masa del planeta Mercurio, dado que el estudio de su órbita presenta discrepancias con los datos observacionales si se realiza directamente desde la mecánica clásica $\left(3.285 \times 10^{23} kg\right)$.
        \item Las rapideces de los cuerpos de interés  \textbf{deben ser pequeñas} comparadas a la rapidez de la luz $\left(299.792,458 km/s\right)$.
        \item  La escala de tiempo en que se realiza el estudio \textbf{debe ser pequeña} en comparación a escalas de tiempo que tiendan a las astronómicas.
    \end{itemize}

    De lo anterior se puede concluir que la Mecánica Clásica está limitada, en el desarrollo que se le dará aquí, al estudio de fenómenos en escalas humanas. Grosso modo, \textbf{la Mecánica Clásica se emplea para estudiar cosas que los seres humanos pueden ``ver'', tanto en tamaño como en escala temporal.} Las comillas al ``ver'' es porqué es posible calcular con suficiente presición algunos fenómenos que ocurren por debajo de las escalas de visión humana con bastante presición.


\end{document}
